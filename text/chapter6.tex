\section{결론}
비록 수렴 속도, 오버슈트 같은 성능 지표 및 복잡한 비선형계에서의 적응형 성능은 최신 ML 기반(시계열, 강화 학습 등등)의 제어기 보다 떨어질 수 있으나 RBF-NN은 은닉층이 얇고 (보통 1층), 각 층의 너비가 좁아 (십여 개) 실시간 학습이 가능할 정도로 학습 및 추론 시간이 매우 빠릅니다.
%
이는 저성능 임베디드 시스템에 충분히 탑재할 수 있는 신경망 기반 적응형 제어기로 100DMIPS의 MCU에서 센서 퓨전, 실시간 학습, 추론 및 제어 출력 계산,  텔레메트리 데이터 인코딩/디코딩과 같은 여러 작업을 수백 Hz의 주기로 실행함에 부족함이 없을 정도입니다.

결론적으로 고전 제어기 기반의 안정성과, 적응형으로 인한 성능 개선 및 플랜트의 의존도 감소의 가능성, 저성능에서도 실행 가능하다는 장점을 가지고 있어 실무에 적용할 대안으로 기대됩니다.
%
현 시점에서 각도 제어만 구현 되어있어 위치제어를 추가적으로 구현해야 하며 제어기의 성능을 더욱더 끌어올리기 위한 추가적인 실험이 필요합니다. SW 아키텍처 부분에서도 텔레메트리 데이터 인/디코딩 방식을 protobuf 같은 진보된 이기종간 프로토콜로 바꾸고 UI를 개선해야 하는 등 추가 과제가 남아있습니다. 

\clearpage